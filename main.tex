\documentclass{article}
\usepackage{amsmath}
\usepackage[margin=2.50cm]{geometry} % Makes the Margins not stupid
\usepackage{amssymb}
\usepackage{amsfonts}
\usepackage[hidelinks]{hyperref}
\usepackage{cancel}
\usepackage[type={CC}, modifier={by}, version=4.0]{doclicense}

\def\ct{\mathbf{c}_{\theta}}
\def\st{\mathbf{s}_{\theta}}
\def\sxx{S_{xx}}
\def\sxy{S_{xy}}
\def\sxz{S_{xz}}
\def\syx{S_{yx}}
\def\syy{S_{yy}}
\def\syz{S_{yz}}
\def\szx{S_{zx}}
\def\szy{S_{zy}}
\def\szz{S_{zz}}

\author{James Wright}

\begin{document}
\title{Converting Tensors from Cartesian to Cylindrical}
\maketitle
\doclicenseThis{}

From \url{http://solidmechanics.org/text/AppendixD/AppendixD.htm} we get:

\begin{equation}
    \begin{bmatrix}
    S_{rr} & S_{r\theta} & S_{rz} \\
    S_{\theta r} & S_{\theta\theta} & S_{\theta z} \\
    S_{zr} & S_{\theta z} & S_{zz}
    \end{bmatrix}
    =
    \begin{bmatrix}
        \cos \theta & \sin \theta & 0\\
        -\sin \theta & \cos \theta & 0 \\
        0 & 0 & 1
    \end{bmatrix}
    \begin{bmatrix}
    S_{xx} & S_{xy} & S_{xz} \\
    S_{yx} & S_{yy} & S_{yz} \\
    S_{zx} & S_{zy} & S_{zz}
    \end{bmatrix}
    \begin{bmatrix}
        \cos \theta & -\sin \theta & 0\\
        \sin \theta & \cos \theta & 0 \\
        0 & 0 & 1
    \end{bmatrix}
\end{equation}

Take the first two matrices and multiply them:

Note: to ease space constraints, the following translations have been defined:

\begin{align}
    \cos \theta &= \ct \\
    \sin \theta &= \st 
\end{align}
%
\begin{multline}
    \begin{bmatrix}
        \cos \theta & \sin \theta & 0\\
        -\sin \theta & \cos \theta & 0 \\
        0 & 0 & 1
    \end{bmatrix}
    \begin{bmatrix}
    S_{xx} & S_{xy} & S_{xz} \\
    S_{yx} & S_{yy} & S_{yz} \\
    S_{zx} & S_{zy} & S_{zz}
    \end{bmatrix}
    \\
    \Longrightarrow
    \begin{bmatrix}
        \ct\sxx + \st\syx + 0 & \ct\sxy + \st\syy + 0 & \ct\sxz + \st\syz + 0 \\
        -\st\sxx + \ct\syx + 0 & -\st\sxy + \ct\syy + 0 & -\st\sxz + \ct\syz + 0 \\
        0 + 0 + \szx & 0 + 0 + \szy & 0 + 0 + \szz
    \end{bmatrix}
    \\
    \Longrightarrow
    \begin{bmatrix}
        \ct\sxx + \st\syx & \ct\sxy + \st\syy & \ct\sxz + \st\syz \\
        -\st\sxx + \ct\syx & -\st\sxy + \ct\syy & -\st\sxz + \ct\syz \\
        \szx & \szy & \szz
    \end{bmatrix}
\end{multline}

Multiply the result by the third matrix:
\begin{multline}
    \begin{bmatrix}
        \ct\sxx + \st\syx & \ct\sxy + \st\syy & \ct\sxz + \st\syz \\
        -\st\sxx + \ct\syx & -\st\sxy + \ct\syy & -\st\sxz + \ct\syz \\
        \szx & \szy & \szz
    \end{bmatrix}
    \begin{bmatrix}
        \cos \theta & -\sin \theta & 0\\
        \sin \theta & \cos \theta & 0 \\
        0 & 0 & 1
    \end{bmatrix}
    = 
    \\
    \begin{bmatrix}
        \ct^2\sxx + \st\ct\syx + \ct\st\sxy + \st^2\syy & 
        -\ct\st\sxx - \st^2\syx + \ct^2\sxy + \ct\st\syy & 
        \ct\sxy + \st\syz \\
        -\st\ct\sxx + \ct^2\syx - \st^2\sxy + \ct\st\syy & \st^2\sxx - \ct\st\syx - \ct\st\sxy + \ct^2\syy & -\st\sxz + \ct\syz \\
        \ct\szx + \st\szy & -\st\szx + \ct\szy & \szz
    \end{bmatrix}
    \label{eq:something}
\end{multline}

Combine like terms, assuming that the tensor is symmetric (ie. \(S_{ij} = S_{ji}\)):

\begin{equation}
    (\ref{eq:something}) \Longrightarrow 
    \begin{bmatrix}
        \ct^2\sxx + 2\ct\st\sxy + \st^2\syy & 
        -\ct\st\sxx + (\ct^2- \st^2)\sxy + \ct\st\syy & 
        \ct\sxz + \st\syz \\
        -\ct\st\sxx + (\ct^2 - \st^2)\sxy + \ct\st\syy & \st^2\sxx - 2\ct\st\sxy + \ct^2\syy & -\st\sxz + \ct\syz \\
        \ct\sxz + \st\syz & -\st\sxz + \ct\syz & \szz
    \end{bmatrix}
\end{equation}

This can be further simplified by combinging the \(\ct\st\) groups:

\begin{equation}
    \begin{bmatrix}
    S_{rr} & S_{r\theta} & S_{rz} \\
    S_{\theta r} & S_{\theta\theta} & S_{\theta z} \\
    S_{zr} & S_{\theta z} & S_{zz}
    \end{bmatrix}
    =
    \begin{bmatrix}
        \ct^2\sxx + 2\ct\st\sxy + \st^2\syy & 
        \ct\st(\syy-\sxx) + (\ct^2- \st^2)\sxy & 
        \ct\sxz + \st\syz \\
        \ct\st(\syy-\sxx) + (\ct^2- \st^2)\sxy & \st^2\sxx - 2\ct\st\sxy + \ct^2\syy & -\st\sxz + \ct\syz \\
        \ct\sxz + \st\syz & -\st\sxz + \ct\syz & \szz
    \end{bmatrix}
\end{equation}

To list off the unique components:

\begin{align}
        S_{rr} &= \ct^2\sxx + 2\ct\st\sxy + \st^2\syy \\
        S_{r\theta} &= \ct\st(\syy-\sxx) + (\ct^2- \st^2)\sxy \\
        S_{rz} &= \ct\sxz + \st\syz \\
        S_{\theta\theta} &= \st^2\sxx - 2\ct\st\sxy + \ct^2\syy \\
        S_{\theta z} &= -\st\sxz + \ct\syz \\
        S_{zz} &= S_{zz}
\end{align}

%%* Result without the function definitions
% \begin{align}
%         S_{rr} &= \mathbf{c}_{\theta}^2S_{xx} + 2\mathbf{c}_{\theta}\mathbf{s}_{\theta}S_{xy} + \mathbf{s}_{\theta}^2S_{yy} \\
%         S_{r\theta} &= \mathbf{c}_{\theta}\mathbf{s}_{\theta}(S_{yy}-S_{xx}) + (\mathbf{c}_{\theta}^2- \mathbf{s}_{\theta}^2)S_{xy} \\
%         S_{rz} &= \mathbf{c}_{\theta}S_{xz} + \mathbf{s}_{\theta}S_{yz} \\
%         S_{\theta\theta} &= \mathbf{s}_{\theta}^2S_{xx} - 2\mathbf{c}_{\theta}\mathbf{s}_{\theta}S_{xy} + \mathbf{c}_{\theta}^2S_{yy} \\
%         S_{\theta z} &= -\mathbf{s}_{\theta}S_{xz} + \mathbf{c}_{\theta}S_{yz} \\
%         S_{zz} &= S_{zz}
% \end{align}

\end{document}
